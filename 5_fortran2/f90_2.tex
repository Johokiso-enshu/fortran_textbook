\chapter{Fortran90の基礎2〜配列$\cdot$繰り返し処理$\cdot$ファイル出力〜}

本章では, 主にFibonacci数列を題材として, Fortran90の様々な機能を学習する. 

\section{Fibonacci数列}
Fibonacchi数列とは次の三項間漸化式で定義される数列である. 
\begin{equation}
\begin{split}
a_0&=0, \\
a_1&=1, \\
a_{n+2}&=a_{n+1}+a_{n} \ \ \ \ (n=0, 1, 2, \cdots). 
\end{split}
\end{equation}

試しに, 最初の数項を書き下すと以下のようになっている. 
\[
0, \ 1, \ 1, \ 2, \ 3, \ 5, \ 8, \ 13, \ 21, \ 34, \ 55, \ 89, \ 144, \ 233, \ 377, \ 610, \ 987, \ 1597, \cdots
\]

なお, 詳細は省くが, Fibonacci数列の一般項は以下の式で与えられることが知られている. 
\begin{equation}
a_n=\frac{1}{\sqrt{5}} \Bigg\{ \Bigg( \frac{1+\sqrt{5}}{2}\Bigg)^n- \Bigg( \frac{1-\sqrt{5}}{2}\Bigg)^n \Bigg\}. 
\end{equation}
$n=0, 1, 2$などを代入して確認してみよ. 

%\begin{equation}
%\lim_{n\to \infty}\frac{a_n}{a_{n-1}}=\frac{1+\sqrt{5}}{2}
%\end{equation}

Fibonacci数列の第$n-1$項目に対する第$n$項の比$\phi_n$は$n\to \infty$の極限で一定値$(1+\sqrt{5})/2$に収束し, 
この値は黄金比として知られている: 
\begin{equation}
\phi_n = \frac{a_n}{a_{n-1}} \to \frac{1+\sqrt{5}}{2} \ \ \ (n \to \infty). 
\end{equation}

Fibonacci数列$a_n$と$\phi_n$を$n=5$まで, まずは素直に計算したのが次のプログラムである. 

\lstinputlisting[caption=Fibonacci数列の計算. , label=fibo1]{5_fortran2/codes/fibo1.f90}


\section{配列}
上のソースコードではFibonacci数列の第5項目までを計算するために, 
\verb|a0, a1, a2, a3, a4, a5|の6つの倍精度実数型変数を宣言した. 
しかしながら, もしもFibonacci数列を$n=100$まで計算しようとすると, 
その全てをひとつひとつ宣言するのは大変である. 
そこで, Fortran90の配列と呼ばれる機能を用いる. 
これは数字の組をひとまとめにして, 一つの変数として保持しておく機能であり, 
スカラーに対するベクトルと考えるとイメージしやすいであろう. 

配列を用いたFibonacci数列の計算プログラムを以下に示す. 
\lstinputlisting[caption=配列を用いたFibonacci数列の計算. , label=fibo2]{5_fortran2/codes/fibo2.f90}

プログラムの3行目が配列の宣言部である. 
\verb|a(0:5)|は\verb|a|という名前の配列を定義し, その引数の下限値が\verb|0|, 上限値が\verb|5|であることを意味する. 
すなわち, $(a_0, a_1, a_2, a_3, a_4, a_5)$をまとめて宣言したことになる. 
$\phi_n$は$n\ge2$で定義すればよいので, \verb|phi(2:5)|としている. 
配列\verb|a|の第0要素を参照するときには, プログラムの6行目のように\verb|a(0)|として, 
括弧の中に指定する. 


\section{繰り返し処理}
上のプログラムの10-12行目, 15-17行目, 20-22行目, 25-27行目は引数が異なるだけで, 
全く同じ処理を繰り返している. 
このような場合, Fortran90のdoループという機能を用いればプログラムを簡潔に書くことができる. 
その例を以下に示す. 
\lstinputlisting[caption=doループを用いたFibonacci数列の計算. , label=fibo3]{5_fortran2/codes/fibo3.f90}
プログラムの9-13行目がdoループによる繰り返し処理である. 
ソースコード\ref{fibo2}と比較してみよ. 
このようにdoループによる繰り返し処理を用いることで, $n=1000$でも$n=10000$でも容易に計算することができる. 





%%%%%%%%%%%%%%%%%%%%%%%%%%%%%%%%%%%%%%%%%%%%%%%%%%%%%%%%%%%%%%%%%%%%%%%%%%%%%%%%%%%%%%%%%%%%%%%%%%%%%%%%%
\if0
\section{繰り返し処理とは}
\subsection*{$<$演習課題$>$}
以下で表されるフィボナッチ数列$a_k$について, $k=1,2, \cdots, 10$の値を画面に表示するプログラムを作成せよ.  \\
\begin{eqnarray}
a_0 &=& 1 \\
a_1 &=& 1 \\
a_{k+2} &=& a_{k+1} + a_k
\end{eqnarray}

以下に解答例を示す. 
この例では, 3回同じ計算をソースコードに記述している. 

\lstinputlisting[caption={繰り返し処理を用いないフィボナッチ数列の計算. }, label=fibonacci]{5_fortran2/codes/Fibonacci.f90}

\section{繰り返し処理}
さらに多くの繰り返しを行いたい場合, その操作を全てソースコードに記述するのは現実的でない. 
多くのプログラム言語には, このような同じ処理の繰り返しを行うための枠組みが用意されている. 
Fortranの場合, {\ttfamily do}ループと呼ばれる仕組みで実現できる. 

以下に, フィボナッチ数列を{\ttfamily do}ループにより計算し, 画面に表示するプログラムを示す. 

\lstinputlisting[caption={繰り返し処理を用いたフィボナッチ数列の計算. }, label=fibonacci_doloop]{5_fortran2/codes/Fibonacci_doloop.f90}
\fi
%%%%%%%%%%%%%%%%%%%%%%%%%%%%%%%%%%%%%%%%%%%%%%%%%%%%%%%%%%%%%%%%%%%%%%%%%%%%%%%%%%%%%%%%%%%%%%%%%%%%%%%%%



\section{ファイル出力とグラフ描画}
以上の例では, 計算された値が画面に出力されるだけなので, その計算結果の様子がつかみにくい. 
計算結果をグラフ化することで, その結果を効果的に理解することが可能となる. 

%計算結果をグラフ化するため, 以下の項目を学習する. 
%\begin{enumerate}
%	\item 計算結果をファイルへ出力する. 
%	\item 結果が格納されたファイルを読み取り, その結果をグラフ化する. 
%\end{enumerate}



\subsection*{ファイル出力}
上のプログラムに少し手を加えて, 計算結果をファイルに出力するように変更したものが以下である.
%\lstinputlisting[caption=フィボナッチ数列を計算し, ファイルに出力するプログラムのソースコード. , label=fileFibonacci]{5_fortran2/codes/Fibonacci_file.f90}
\lstinputlisting[caption=Fibonacci数列のファイル出力. , label=fibo4]{5_fortran2/codes/fibo4.f90}

open文を用いて装置番号10のファイル``output.dat''を開いておき,
write文で書き出す際には装置番号10を指定している
\footnote{
なお, 装置番号は5(標準入力)と6(標準出力)を除く1から99までの任意の数を指定することができる.
}.
プログラムを実行し, ``output.dat''というファイルが作成され, その中に
フィボナッチ数列の計算結果が50個出力されていることを確認せよ. 


\subsection*{グラフ描画}
上記のように計算結果をファイルに出力しておくと, 以前学習したgnuplotなどを用いて描画することができる. 
\begin{Verbatim}[frame=single]
>gnuplot
\end{Verbatim}
でgnuplotを起動したのち, 
\begin{Verbatim}[frame=single]
gnuplot> plot 'output.dat' w l
\end{Verbatim}
などとするとFibonacci数列が描画できる. 

\subsection*{$<$演習課題2.1$>$}
Fibonacci数列$a_n$や数列の比$\phi_n$をグラフ化せよ. 
必要に応じて対数グラフなどを利用し, 見やすいグラフにすること. 
グラフはpng形式で出力せよ.  

\subsection*{$<$演習課題2.2$>$}
上に示したソースコードでは, \verb|a(0:50)|のように長さ51の配列を定義しているが, 
\verb|a(i)|の計算に必要なのは\verb|a(i-1)|, \verb|a(i-2)|だけであり, 
それ以前の数列の値をメモリ上に保存しておく必要はない. 
(例えば, 第100億項まで計算するとき, $10^{10} \times 8\mbox{ Byte}=80\mbox{ GByte}$のメモリが必要となる. 
普通のコンピュータはそこまで大きなメモリをもっていない!)

\section{総和の計算}
繰り返し処理の応用例として, 本節では総和の計算法を取り扱う. 
以下のソースコードは, 
\begin{equation}
S=\sum_{i=1}^{10} i
\label{eq_sum}
\end{equation}
を計算するプログラムである. 

\lstinputlisting[caption={総和の計算. }, label=summation]{5_fortran2/codes/Summation.f90}
このプログラムでどうして(\ref{eq_sum})の総和が計算できるのか, 
doループを書き下すことによって各自確かめてみよ. 
数値計算において総和を計算する場面はしばしば現れるので, 
よく覚えておくこと. 


\subsection*{$<$演習課題2.3$>$}
$N=20$に対して, 
\begin{align}
&\sum_{n=1}^{N} n^2 \\
%&\sum_{n=1}^{N} \frac{1}{n} \\
&\sum_{n=1}^{N} \frac{1}{n^2}
\end{align}
を計算し, 画面表示させるプログラムを作成せよ. 

また, 
\begin{equation}
\sum_{n=1}^{\infty} \frac{1}{n^2}=\frac{\pi^2}{6}
\end{equation}
であることが知られている(ゼータ関数の特殊値). 
$N=1, 2, \cdots$に対する結果をグラフ化し, 厳密解へと漸近する様子を確認せよ. 
また, 厳密解に対する相対誤差は$N$の何乗に比例して減少するか, 両対数グラフを書くことで調べよ. 

\subsection*{$<$演習課題2.4$>$}
階乗
\begin{eqnarray}
k! = \prod_{n=1}^{k} n
\end{eqnarray}
を計算するプログラムを作成せよ. 

% TODO この部分でオーバーフローについても教えれる. 教えても良いかも
階乗$k!$は$k$が大きくなるとすぐに発散してしまう. 
その対数値である$\log(k!) = \sum_{n=1}^{k} \log(n)$ という数列の値を, 
$k=1,2, \cdots, 20$について画面表示させるプログラムを作成せよ. \\
\newline




\subsection*{$<$演習課題2.5$>$}
大きな整数$k$の階乗は, $k!\approx \sqrt{2\pi k}\left(k/e\right)^k$で近似できることが知られている
(スターリングの近似). 

$k=1,2, \cdots, 200$について, $\log(k!)$および$\log(\sqrt{2\pi k}\left(k/e\right)^k)=\frac{1}{2}\log(2\pi k) + k\log(k)-k$を一つのグラフ上に描画し, 
スターリングの公式の有用性を確かめよ. 
\newline
