\chapter{演習課題}
これまで数週にわたってFortranプログラミングを学習してきたが, 
単にプログラムを書いて, 出力された結果をもとに図を描いただけでは社会に対するアウトプットとならない. 
解くべき問題の背景, 問題へのアプローチの仕方, 得られた結果とそれがもつ意味, さらなる問題提起, について
日本語(あるいは英語)でまとめる技術は卒業までに身につけなければならないスキルである. 

\section{科学技術レポート}
これまでのプログラミング課題の中から好きなものを一つ取り上げ, Wordを用いてレポートとしてまとめよ. 
レポートには次の内容を含めること. (各章の題目は以下のものに限定しないが, 参考までに列挙している)

\begin{enumerate}
\item 序論, 緒言, イントロダクション, はじめに, 背景

選んだ問題の背景について述べ, 本レポートにて解くべき問題と目的について述べる. 

\item 数値計算法, 数値解析手法, 数値シミュレーション, アルゴリズム

設定した問題を解くための手法について述べる. 
基礎方程式やアルゴリズム, 設定したパラメータ等についてまとめる. 
必要に応じて, 問題のポンチ絵やプログラムの流れを示すフローチャートを用いる. 
第三者であってもレポートを読めば, 数値計算結果を再現できるように詳細に記述する. 


\item 結果, 数値計算結果, 結果と考察

得られた結果を図や表を用いて示す. 
単に図表を並べるだけではなく, それぞれの図表が意味するものを言葉を用いて丁寧に記述する. 
また, 図には縦軸横軸のラベル, 凡例, キャプション等を必ずつけること. 
有次元量には必ず単位が必要であることに注意せよ. 

なお, 得られた結果の妥当性についてもここで言及する. 

\item 結論, 結言, まとめ, おわりに

レポートの内容を簡潔にまとめ, さらなる問題提起(現状の解析の問題点など)をおこなう. 

\item 参考文献

参考にした文献がある場合には, 必ず参考文献リストを付ける. 
通常は教科書や論文を列挙するが, 
本レポートに関してはWebページ上の情報であってもよい. 
その場合, URLと最終アクセス日時を記すこと. 

\end{enumerate}

\section{科学技術プレゼンテーション}
Wordでまとめたものと同じテーマについて, PowerPointを用いてまとめよ. 
スライドには少なくとも1ページずつ次の内容を含め, 全部で5-10ページ程度にまとめること. 

\begin{enumerate}
\item 表紙

\item 序論, 緒言, イントロダクション, はじめに, 背景

\item 数値計算法, 数値解析手法, 数値シミュレーション, アルゴリズム

\item 結果, 数値計算結果, 結果と考察

\item 結論, 結言, まとめ, おわりに


\end{enumerate}
