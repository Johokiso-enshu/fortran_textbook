\chapter{Fortran90の基礎3〜プログラムフローの構成〜}

この章では, プログラムの条件分岐とサブルーチンを用いたプログラムのカプセル化について学ぶ.
具体例として, 高精度かつ効率的な$\log(k!)$の値の計算プログラムの作成を取り扱う.
%具体的には, ユーザーにより与えられた$k$の値に応じて, $\log(k!)$の値を計算し, 画面に表示することを目指す.

$k$の値が小さいとき, $\log(k!)$の値は, $\sum_{n=1}^{k}\log(k)$からすぐに計算できる.
一方で, $k$の値が例えば100,000を超えるようなときは, 100,000回以上の対数の計算をする必要があり,
大きな計算コストが必要である.
($k!$を計算してから対数をとろうと思っても, $k!$自体が計算機上で発散してしまうのでできない.)
そこで前章の演習問題で学んだように,
大きい$k$の場合には直接$\sum_{n=1}^{k}\log(k)$を計算するよりも,
その近似である$k\log{k}-k$を計算する方が圧倒的に高速である.

高精度・高効率に$\log(k!)$を計算するプログラムは以下のような手順を実行することになる.

\begin{enumerate}
  \item ユーザーから$k$の値を受け取る.
  \item 与えられた$k$の値と, あるしきい値$k_0$の大小関係を判定する.
  \begin{itemize}
    \item $k \leq k_0$のとき, $\sum_{n=1}^{k}\log(k)$を計算する.
    \item $k>k_0$のとき, $k\log{k}-k$を計算する.
  \end{itemize}
 \item 結果を画面に表示する.
\end{enumerate}


\subsection*{$<$演習課題3.1$>$}
$\sum_{n=1}^{k}\log(k)$の値と$\frac{1}{2}\log(2\pi k) + k\log{k}-k$の値の差の絶対値を
$k=1,2, \cdots, 1000$について計算し, その差が$10^{-4}$より小さくなる最小の$k$の値を求めよ. 
なお, 本課題は以下の``条件分岐''を学んでから取り組むと, 効率的に解決できる.
% answer: lfact_k.f90


\section{条件分岐}
上に示した手順のうち, 2では, 特定の条件が満たされたときに計算の挙動を変化させている.
それを実現する枠組みを{\bfseries 条件分岐}と呼ぶ.
Fortranでは, {\ttfamily if}文を用いて条件分岐を行う.

以下に, {\ttfamily if}文の一つの用例を示す.
ここでは, 二次方程式の判別式を計算して, それが正, 零, 負であるときにそれぞれ異なる処理を行っている.
\lstinputlisting[caption={判別式による二次方程式の解の判定. }, label=ifstatement2]{6_fortran3/codes/IfStatement2.f90}


このように, {\ttfamily if}文を使えば, 括弧内の条件が正しいか正しくないかに応じて, 処理を分岐することができる.
具体的には以下の表のように, 適当な等式, あるいは不等式を書くこと.

\begin{table}[h]
  \caption{等式, 不等式の書き方. }
  \begin{center}
    \begin{tabular}{ccc}
      \hline
      Fortranでの書き方   & \multicolumn{2}{c}{意味} \\ \hline
      \verb|x == y|   & x $=$ y &(イコール)\\
      \verb|x /= y|   & x $\ne$ y &(ノットイコール)\\
      \verb|x < y|    & x $<$ y &(小なり)\\
      \verb|x <= y|   & x $\le$ y &(小なりイコール)\\
      \verb|x > y|    & x $<$ y &(大なり)\\
      \verb|x >= y|   & x $\ge$ y &(大なりイコール)\\ \hline
    \end{tabular}
  \end{center}
\end{table}

\verb|if|文の使用パターンは大きく分けて以下の4通りである.
\begin{Verbatim}[frame=single]
  if (条件) 処理
\end{Verbatim}
\begin{Verbatim}[frame=single]
  if (条件) then
    条件が真の場合の処理
  endif
\end{Verbatim}
\begin{Verbatim}[frame=single]
  if (条件) then
    条件が真の場合の処理
  else
    条件が偽の場合の処理
  endif
\end{Verbatim}
\begin{Verbatim}[frame=single]
  if (条件1) then
    条件1が真の場合の処理
  elseif (条件2)
    条件1が偽で, 条件2が真の場合の処理
  else
    条件1, 2が偽の場合の処理
  endif
\end{Verbatim}




\subsection*{$<$演習課題3.2$>$}
二次方程式の判別式を計算し, それが正のときは二実数解,
零のときは重解, 負のときは複素数解を計算し, 出力するプログラムを作成せよ.


\subsection*{$<$演習課題3.3$>$}
以前の演習課題で求めた
$\sum_{n=1}^{k}\log(k)$の値と$\frac{1}{2}\log(2\pi k) + k\log{k}-k$の値の
差の絶対値が$10^{-4}$より小さくなる最小の$k$の値を用い,
効率よく$\log(k!)$を計算できるプログラムを作成せよ.



\section{カプセル化}
この節では, 大規模なプログラミングを行う上で重要な概念となる
``カプセル化'' について学ぶ.
カプセル化とは, 小さな問題を解決する機能を独立させそのインターフェースを提供することである.
これにより複雑な問題を複数の小さな問題に分割することができ, その論理フローを明瞭にすることができる.
\newline

例えば, これまで効率よく$\log(k!)$を計算できるプログラムを作成してきた.
そのために, $k!$が$k$が大きい領域でどのように振る舞うかや,
十分に精度よく近似できる$k$の値などを詳細に調べた.

しかし, 別の(より大規模な)プログラムで$\log(k!)$の計算値を用いたい場合,
このようなことを再度考え, 新たにプログラムを書くことは非効率である.
``$\log(k!)$を計算する''という機能を独立させ, そのインターフェースを提供することで,
次回からは$\log(k!)$の計算方法に頭を使う必要はなくなり, 別のことに集中することができる.


次では,
このようなカプセル化を実現するための一つの枠組みである
%``関数''や
``サブルーチン''を学ぶ.


\subsection{サブルーチン}
これまで作成してきた$\log(k!)$の計算プログラムを
サブルーチンを用いて書き換えると以下のようになる. \\

%TODO プログラムを作る!!
\lstinputlisting[caption={サブルーチンを用いた$\log(k!)$計算プログラムの例. }, label=lgamma_subroutine]{6_fortran3/codes/lgamma_subroutine.f90}

上記のようにサブルーチンを定義しておけば, {\ttfamily LogFactorial} (Factorialとは階乗のこと)を呼び出すだけで,
何度でも$\log(k!)$の値を計算することができる.

サブルーチンの記述方法は以下の通りである.
\begin{Verbatim}[frame=single]
  subroutine サブルーチン名(仮引数1,仮引数2,・・・)
  宣言部
  命令
  ・・・
  return
  end subroutine サブルーチン名
\end{Verbatim}

\begin{enumerate}
\item サブルーチン名\\
サブルーチン名の付け方は通常の変数名と同じである.
\item 仮引数1,仮引数2,・・・\\
サブルーチン中で使用する仮の引数. この定義の中だけで有効なため,
プログラム中で使用している関数名も使用することができる.
\item 宣言部\\
サブルーチンおよびサブルーチン中で使用する変数(引数, 戻り値含む)の型宣言が必要である.
%入力用引数には「INTENT (IN)」を, 出力用引数には「INTENT (OUT)」を, 入出力両用引数には「INTENT (INOUT)」を指定する.
\item 命令\\
複数の引数について処理を行っても良いし, 値をまったく返さず, 何らかのタスクを行うだけの内容でも良い. \\
\end{enumerate}


\begin{comment}
\subsection*{$<$演習課題$>$}
フィボナッチ数列は大きい$k$に関して,
$\log(a_k) \approx k\log\left(\frac{1+\sqrt{5}}{2}\right)$
と近似できることが知られている.

$\log(k!)$ と同様に, 任意の$k$に関して, $\log(a_k)$の値を精度よく高効率に計算できる
機能をサブルーチンとして作成せよ.
\end{comment}


\begin{comment}
以下は二次方程式の判別式を計算して, それが正, 零, 負であるときにそれぞれ異なる処理をするプログラムである.
\lstinputlisting[caption={判別式による二次方程式の解の判定. }, label=ifstatement2]{6_fortran3/codes/IfStatement2.f90}

\subsection*{$<$演習課題$>$}
二次方程式の判別式を計算し, それが正のときは二実数解,
零のときは重解, 負のときは複素数解を計算するプログラムを作成せよ.


\subsection*{$<$演習課題$>$}
自然対数の底$e$の近似値を以下に示す二つの数列を用いて計算せよ.
$n=1, 2, \cdots$と大きくしていき, 真値へと漸近する様子を確認せよ.
また, 真値との相対誤差が$10^{-10}$以下になったときにプログラムを停止するようにせよ.
\begin{equation}
a_n= \Big( 1+\frac{1}{n}\Big)^n
\end{equation}
\begin{equation}
b_n=\sum_{m=0}^{n}\frac{1}{m!}
\end{equation}

以下に, ソースコードの一部をヒントとして示す. 必要であれば参考にして良い.
%TODO a_n を求めるコードのヒントを追加する.

%\begin{equation}
%e=\lim_{n\to \infty} a_n = \lim_{n\to \infty} b_n.
%\end{equation}

%TODO もっと演習課題を追加する.
\end{comment}


\subsection{行列演算}
%この章では, 複数のデータを同時に扱うための``配列''について学ぶ.
%配列は一連のデータの集合である.
%例えば3次元空間内のベクトル$\mathrm{x}$は,
%$\mathrm{x}=\{x, y, z\}$という3要素を持つ配列として表すことができる.

%以下では, 線形代数の諸操作をFortranを用いて実装する.

この節では, 上で学んだカプセル化に親しむため, 行列の演算を題材とする.
次の例は, 3要素を持つベクトル$\bm{x}, \bm{y}$の内積を計算するプログラムである.

\lstinputlisting[caption={ベクトルの内積. }, label=dotproduct]{7_fortran4/codes//Subroutine0.f90}

%%%%%%%%%%%%%%%%%%%%%%%%%%%%%%%%%%%%%%%%%%%%%%%%%%%%%%%%%%%%%%%%%%
\if0
[ソースコードの解説]\\
配列変数をFortran内で使用するためには, 宣言時にその要素数を明示する必要がある.

\begin{Verbatim}[frame=single]
> double precision x(3)
\end{Verbatim}
これは, 変数{\ttfamily x}が, 倍精度の要素を3つ持つ配列であることを示している.

配列変数に値を代入するためには, その何番目の要素に値を代入するかをカッコ内に示す.
\begin{Verbatim}[frame=single]
> x(1) = 1.d0
\end{Verbatim}
これは, {\ttfamily x}の1番めの要素に1を代入する操作を示している.

配列変数は, 以下のように{\ttfamily do}ループを用いることで効率的に操作することができる.
\begin{Verbatim}[frame=single]
> do i = 1, 3
>   dot = dot + x(i) * y(i)
\end{Verbatim}
\fi
%%%%%%%%%%%%%%%%%%%%%%%%%%%%%%%%%%%%%%%%%%%%%%%%%%%%%%%%%%%%%%%%%%

\subsection*{$<$演習課題3.4$>$}
上記ソースコードを参考にし, 以下の2つの行列$X$および$Y$の
積を計算するプログラムを作成せよ.
\begin{equation}
X =
\begin{bmatrix}
  1.0 & 2.2 & 3.2 \\
  2.1 & 2.3 & 4.1 \\
  5.2 & 6.2 & -3.0\\
\end{bmatrix},\\
Y =
\begin{bmatrix}
  1.1 & 3.2 &  5.2 \\
  2.4 & 3.3 &  2.1 \\
  5.3 & 4.2 & -3.0 \\
\end{bmatrix}
\end{equation}
なお, 配列\verb|x|と\verb|y|は次のように二次元配列として宣言すること.
\verb|double precision:: x(3:3), y(3:3)|\\


上記の内積計算プログラムをサブルーチン化したものが以下である.

% TODO ソースコードを加える.
\lstinputlisting[caption={ベクトルの内積. }, label=subroutine]{7_fortran4/codes//Subroutine.f90}

\subsection*{$<$演習課題3.5$>$}
2つの行列の積を計算するサブルーチンを作成し, 適当な行列に対して実行せよ.

\subsection*{$<$演習課題3.6$>$}
2行2列の正方行列$\mathrm{A}$の逆行列を計算するサブルーチンを作成せよ.
なお,
$\mathrm{A} =
\begin{bmatrix}
  a_{1,1} & a_{1,2} \\
  a_{2,1} & a_{2,2} \\
\end{bmatrix}
$\\
の逆行列$\mathrm{A}^{-1}$は以下のように表される.
\begin{equation}
\mathrm{A}^{-1} = \frac{1}{a_{1,1}a_{2,2}-a_{1,2}a_{2,1}}
\begin{bmatrix}
  a_{2,2} & -a_{1,2}\\
  -a_{2,1} & a_{1,1}
\end{bmatrix}
\end{equation}


\begin{comment}
  \section{配列}
  %\lstinputlisting[caption={ベクトルの内積. }, label=dotproduct]{7_fortran4/codes//Subroutine0.f90}

  \lstinputlisting[caption={行列の和. }, label=matrixsum]{7_fortran4/codes//MatrixSum.f90}

  \subsection*{$<$演習課題$>$}
  適当な数列を読み込み, その平均値と標準偏差を計算するプログラムを作成せよ.

  \section{サブルーチン}
  ソースコード\ref{dotproduct}のうち,
  内積計算のように汎用性のある部分は独立した副プログラム(サブルーチン)として分けて記述するのがよい.
  以下にサブルーチンDotProductとそれを呼び出すメインプログラムの例を示す.
  %\lstinputlisting[caption={ベクトルの内積. }, label=subroutine]{7_fortran4/codes//Subroutine.f90}
  上の例のように数十行程度の短いプログラムの場合にはあえてサブルーチン化する利点は感じられないかもしれないが,
  同じ処理を繰り返し実施する場合や,
  数千, 数万行にも及ぶ長いプログラムを作成する場合にはサブルーチン化は必須の技術となる.


  \subsection*{$<$演習課題$>$}
  三次元ベクトル$\bm{a}, \bm{b}$の成分を入力し, それらの外積
  \begin{equation}
  \begin{pmatrix}
  a_1 \\ a_2 \\ a_3
  \end{pmatrix}
  \times
  \begin{pmatrix}
  b_1 \\ b_2 \\ b_3
  \end{pmatrix}
  =
  \begin{pmatrix}
  a_2b_3-a_3b_2 \\ a_3b_1-a_1b_3 \\ a_1b_2-a_2b_1
  \end{pmatrix}
  \end{equation}
  を計算するプログラムを作成せよ.
  外積の計算部分はサブルーチン化せよ.
  %\lstinputlisting[caption={繰り返し処理. }, label=doloop]{7_fortran4/codes//CrossProduct.f90}

  \subsection*{$<$演習課題$>$}
  三次元ベクトル$\bm{a}, \bm{b}, \bm{c}$の成分を入力し, ベクトル三重積
  \begin{equation}
  (\bm{a} \times \bm{b}) \times \bm{c}
  \end{equation}
  を計算するプログラムを作成せよ.
  また, ベクトル三重積の公式,
  \begin{equation}
  (\bm{a} \times \bm{b}) \times \bm{c} = (\bm{a} \cdot \bm{c})\bm{b} - (\bm{b} \cdot \bm{c})\bm{a}
  \end{equation}
  の右辺を計算し, 等号が成り立っていることを確認せよ.
  %\lstinputlisting[caption={繰り返し処理. }, label=doloop]{7_fortran4/codes//VectorTripleProduct.f90}

  %\lstinputlisting[caption={繰り返し処理. }, label=doloop]{7_fortran4/codes//Inverse2D.f90}
  %\lstinputlisting[caption={繰り返し処理. }, label=doloop]{7_fortran4/codes//MatrixVector.f90}
  %\lstinputlisting[caption={繰り返し処理. }, label=doloop]{7_fortran4/codes//MatrixProduct.f90}


  \section*{その他書きたいこと}
  \begin{enumerate}
  \item parameter文
  \item exit文(doループを抜けるとき)
  \item goto文(多重ループを抜けるときだけ)
  \item 継続行の書き方
  \item 無限ループの書き方
  \item gnuplotで絵が描けるような課題を増やしたい
  \end{enumerate}

  \chapter{数値計算への応用}
  \section{漸化式}
  \begin{enumerate}
  \item 等差数列
  \item 階差数列
  \end{enumerate}
  など簡単な漸化式のサンプルプログラムと演習問題.


\end{comment}
